\section{Testprotokoll vom XX.06.2017}
\label{sec: Testprotokoll vom XX.06.2017}

\begin{table}[H]
\begin{small}
	\begin{center}
  	\caption{Testprotokoll vom XX.06.2017}
   	\renewcommand{\arraystretch}{1.2}
    \begin{tabularx}{\textwidth}{|X|X|L{2cm}|}	
    \hline
    				
    \textbf{Vorgehen} & \textbf{Soll} & \textbf{Ist} \\ \hline
    
    Aufruf der URL \texttt{XXX} & 
    Anzeige des Login-Feldes für das Admin-Frontend &
     \\ \hline
    
    Anmeldung mit dem Benutzernamen X und dem Passwort Y & 
    Erfolgreicher Login, Anzeige des Warenwirtschaft-Frontend &
     \\ \hline
    
    - & 
    Das Frontend zeigt eine Liste der Waren pro Produkt mit ihrem Zeitstempel &
     \\ \hline
     
     - &
     Es wird angezeigt, dass keine Kirschen mehr vorhanden sind & 
    \\ \hline
    
    Hinzufügen von 5 Kiwis über den Button \texttt{+} &
    Es werden 5 Kiwis mit der aktuellen Zeit zu dem Bestand hinzugefügt &
    \\ \hline
    
    Entfernen von 2 Tomaten über den Button \texttt{-} &
    Es werden die angegeben 2 Tomanten aus dem Bestand entfernt & 
    \\ \hline
    
    Aufruf der URL \texttt{XXX} &
    Anzeige des Kunden-Frontend für Kirschen,
    Anzeige der Verfügbarkeit mit einer roten Ampel &
    \\ \hline
    
    Aufruf der URL \texttt{XXX} &
    Anzeige des Kunden-Frontend für Tomaten,
    Anzeige der Verfügbarkeit mit einer gelben Ampel &
	\\ \hline
    
    Aufruf der URL \texttt{XXX} &
    Anzeige des Kunden-Frontend für Kiwis,
    Anzeige der Verfügbarkeit mit einer grünen Ampel &
	\\ \hline
	
	Bestellen von 2 Kiwis &
	Im Admin-Frontend werden 2 Kiwis als blockiert angezeigt &
	\\ \hline
	
	Transportbestätigung für die 2 Kiwis &
	Im Admin-Frontend werden die 2 Kiwis automatisch aus dem Bestand gelöscht & 
	\\ \hline
	
	\end{tabularx}
	\label{tabl:Testprotokoll vom XX.06.2017}
	\end{center}
\end{small}
\end{table}

 