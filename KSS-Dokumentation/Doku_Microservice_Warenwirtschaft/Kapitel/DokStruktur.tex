\section{Dokumentationsstruktur}
\label{sec: Dokumentationsstruktur}

Für die Dokumentation des Microservice Warenwirtschaft wurden eine Kombination aus zwei Dokumenten gewählt. Zum einen beschreibt ein Handout auf einer Seite die Funktionen des Admin-Frontends für den Benutzer. Diese sehr kurze Dokumentenform wurde gewählt, da Benutzer häufig nicht gewillt sind, umfangreiche Anleitungen zu lesen um eine Anwendung nutzen zu können. Vielmehr wollen sie schnell einen Überblick der Kernfunktionalitäten erhalten. Zu diesem Zweck arbeitet das Handout mit einem aufbereiteten Screenshot des Admin-Frontends und einer Beschreibung der Funktionen in Stichpunkten. \par 
Auf der anderen Seite muss der Microservie auch für Entwickler dokumentiert sein, hierfür wurde diese Dokumentation angelegt. Sie beginn anstelle eines Abstract mit einem Steckbrief des Microservice, der dessen grundlegende Struktur und Funktionalität kurz beschreibt. In dem eigentlichen Dokument werden dann zunächst die Anforderungen an den Microservice beschrieben, da ihre Umsetzung das primäre Ziel der Entwicklung ist. Weiter werden der Microservice mit seinem Aufbau, den Schnittstellen und der Anpassung des gegebenen Monolithen beschrieben. Es folgt die Dokumentation von Implementierungsregeln und der gewählten Blackbox-Testfälle. Der Anhang dieser Dokumentation umfasst einen Anleitung für den Start des Microservice, das Handout des Admin-Frontends sowie ein beispielhaftes Testprotokoll. Auf Details wir ein Abkürzungs- oder Literaturverzeichnis wurde in dieser Dokumention bewusst verzichtet, um sie kurz zu halten. Zitate und Verweise werden hier in Form von Fußnoten integriert.