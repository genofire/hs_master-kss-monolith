
\documentclass[11pt, a4paper]{article}

\usepackage[ngerman]{babel}					% Deutsches Wörterbuch und Zeilenumbrüche
\usepackage{blindtext}						% Testtext
\usepackage[T1]{fontenc}					% Ausgabe encoding
\usepackage[utf8]{inputenc}					% Eingabe encoding
\usepackage[colorinlistoftodos]{todonotes}
\usepackage{xcolor}							% Farben
\usepackage{framed, color}					% Hintergrundfarben für Boxen
\usepackage{textcomp}						% Sonderzeichen
\usepackage{amsmath}     
\usepackage[automark,headsepline]{scrpage2} % Kopf- und Fußzeilen
\usepackage{amsmath,amsfonts,amssymb}		% Mathematik
\usepackage{graphicx}						% Bilder einfügen
\usepackage{wrapfig}	   					% Bilder von Text umfließen lassen
\usepackage{float}							% Floatobjekte (u.a. figures und tables)
\usepackage{placeins}      					% FloatBarrier
\usepackage[hyper]{listings}	   			% Schöne Listings
\usepackage[hidelinks]{hyperref}  			% Hyperlinks
\usepackage{prettyref}   				    % Schönere Hyperlinks
\usepackage{multicol}     					% Mehrspaltige Ausrichtung
\usepackage{titlesec}    					% Aussehen der Überschriften
\usepackage{titletoc}						% Aussehen der Überschriften
\usepackage{geometry}						% Außenränder
\usepackage[scaled]{helvet}					% Schriftart Helvetica
\usepackage{courier}						% Schriftart Courier
\usepackage{enumerate}						% Nummerierung
\usepackage{enumitem}						% Nummerierung
\usepackage{tabularx}						% Tabellen
\usepackage{censor}							% SChwärzen von Informationen

% ---- Farben -----------------------------------------------------------------------------------------------
\definecolor{listings}{RGB}{226,226,226}

\definecolor{hs-blau}{RGB}{10,85,140}
\definecolor{hs-tuerkis}{RGB}{50,180,200}
\definecolor{hs-rot}{RGB}{195,5,50}


% ---- Geometrie und Schrift --------------------------------------------------------------------------------
\geometry{verbose,a4paper,tmargin=2.5cm,bmargin=2cm,lmargin=2.5cm,rmargin=2.5cm,headsep=1.5cm}

\renewcommand*\familydefault{ppl} 
\linespread{1.2} 


\titleformat*{\section}{\LARGE}
\titleformat*{\subsection}{\Large}
\titleformat*{\subsubsection}{\large}

\setlength{\abovecaptionskip}{1pt}
\setlength{\belowcaptionskip}{1pt}
\setlength{\parindent}{0pt}
\setlength{\parskip}{5.5pt}


% ---- 1. Kapitelebene ---------------------------------------------------------------------------------------
\titleclass{\subsubsubsection}{straight}[\subsection]

\newcounter{subsubsubsection}[subsubsection]
\renewcommand\thesubsubsubsection{\thesubsubsection.\arabic{subsubsubsection}}

\titleformat{\subsubsubsection} {\large}{\thesubsubsubsection}{1em}{}
\titlespacing*{\subsubsubsection}{0pt}{15pt}{6pt}

\makeatletter

\def\toclevel@subsubsubsection{4}
\def\toclevel@paragraph{5}
\def\toclevel@paragraph{6}
\def\l@subsubsubsection{\@dottedtocline{4}{7em}{4em}}
\def\l@paragraph{\@dottedtocline{5}{10em}{5em}}
\def\l@subparagraph{\@dottedtocline{6}{14em}{6em}}

\makeatother

\setcounter{secnumdepth}{4}
\setcounter{tocdepth}{4}
 

% ---- Listings ----------------------------------------------------------------------------------------------
\lstset{
basicstyle=\ttfamily\color{black}\small,
backgroundcolor=\color{veryverylightgray},
keywordstyle=\color{black}, 
identifierstyle=\color{black}, 
commentstyle=\color{black}, 
showspaces=false,
showtabs=false, 
breaklines=true,
xleftmargin=0pt,
xrightmargin=0pt,
}


% ---- Tabellen, Abbildungen und Aufzählungen -------------------------------------------------------------------------------
\newcolumntype{L}[1]{>{\raggedright\arraybackslash}p{#1}}  % Blocksatz linksbündig
\newcolumntype{C}[1]{>{\centering\arraybackslash}p{#1}}    % Blocksatz zentriert
\newcolumntype{R}[1]{>{\raggedleft\arraybackslash}p{#1}}   % Blocksatz rechtsbündig

  
\setlist[itemize]{leftmargin=*}
\renewcommand*\labelitemi{$-$}
\setlist[1]{itemsep=-0.5pt}
   % \hyphenation{}	Eigene Worttrennung


%------------------------------------------------------------------------------------------------------------
% ---- Definitionen -----------------------------------------------------------------------------------------
%------------------------------------------------------------------------------------------------------------
\title{KSS - Nachhaltige Softwarearchitektur komplexer Websites}
\author{Studenten der Hochschule Bremen}




% ---- Kopf- Fußzeile -----------------------------------------------------------------------------------------

%i - innen, c - mitte, o - außen
\ihead{ \normalfont Admin-Frontend Warenwitschaft}
\chead{}
\ohead{\includegraphics[width=3cm]{./hs-logo.png}}

\ifoot{\normalfont \censor{M.Labusch}, \censor{M.Müller}}
\cfoot{}
\ofoot{\normalfont 1/1}

\pagestyle{scrheadings}  % eigener Stil für alle Seiten

% ---- PDF-Eigenschaften ----------------------------------------------------------------------------------------
\hypersetup{
	pdftitle = {Dokumentation - Microservice Warenwirtschaft},
	pdfsubject = {Nachhaltige Softwarearchitektur komplexer Websites},
	pdfauthor = {Studenten der Hochschule Bremen},
	pdfkeywords = {Hochschule Bremen, Masterstudiengang Komplexe Softwaresysteme, Softwarearchitektur, Nachhaltigkeit, Docker, Go}
}
