\section{Dokumentationsstruktur}
\label{sec: Dokumentationsstruktur}

Für die Dokumentation des Microservice Warenwirtschaft wurden eine Kombination aus drei Dokumenten gewählt. Zum einen beschreibt ein bebildertes Handout auf einer Seite die Funktionen des Admin-Frontends für dessen Benutzer. Diese sehr kurze Dokumentenform wurde gewählt, da Benutzer häufig nicht gewillt sind umfangreiche Anleitungen zu lesen um eine Anwendung nutzen zu können. Vielmehr wollen sie schnell einen Überblick der Kernfunktionalitäten erhalten. Aus diesem Grund wurde auch auf eine Benutzerdokumentation für die Kunden des Webshops Mosh verzichtet, da anzunehmen ist dass die Darstellung der Produktverfügbarkeit als Ampel sich selbst erklärt.\par 
Auf der anderen Seite muss der Microservice auch für die Entwickler dokumentiert sein, hierfür wurde diese Dokumentation angelegt. Sie beginn anstelle eines Abstract mit einem Steckbrief des Microservice, der dessen grundlegende Struktur und Funktionalität kurz beschreibt. Das eigentliche Dokument beschreibt zunächst die Anforderungen an den Microservice, da die Entwicklung sich primär an diesen orientiert. Weiter werden der Microservice mit seinem Aufbau, den Schnittstellen und der Anpassung des gegebenen Monolithen sowie Implementierungsregeln beschrieben. Dieses Dokument schließt mit einem \textit{Getting Started} Guide. Auf Details wir ein Abkürzungs- oder Literaturverzeichnis wurde in dieser Dokumention bewusst verzichtet, um sie kurz zu halten. Zitate und Verweise werden hier in Form von Fußnoten integriert.\par 
Abschließend dokumentiert das Testprotokoll -- als drittes Dokument -- die für diesen Microservice angewendeten Black-Box-Testfälle. Das heißt es umfasst solche Tests die anhand der Anforderungen und aus Sicht des Benutzers durchgeführt wurden.