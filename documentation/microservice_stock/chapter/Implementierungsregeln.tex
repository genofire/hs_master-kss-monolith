\section{Implementierungsregeln}
\label{sec: Implementierungsregeln}
Die folgende Aufzählung gibt einige Regeln für die Implementierung des Microservice Warenwirtschaft vor. Diese sollten im Rahmen einer Weiterentwicklung eingehalten werden, um die Konsistenz des Codes aufrecht zu erhalten. 

\begin{enumerate}
	\item Packages werden eindeutig und sprechend benannt
	\item Go-Files werden eindeutig und sprechend benannt
	\item Wenn ein Package nur ein Go-File enthält, erhält dieses den Namen seines Packages
	\item Vor jedem Package steht ein ein- bis zweizeiliger, beschreibender Kommentar, der die Hauptfunktionalitäten wiedergibt
	\item Vor jeder Funktion steht ein zwei- bis dreizeiliger, beschreibender Kommentar, dieser enthält
	\begin{enumerate}
		\item eine ein- bis zweizeilige Beschreibung der Funktionalität
		\item eine einzeilige Beschreibung der Eingabe- und Rückgabewerte (entfällt, wenn diese nicht vorhanden sind)
	\end{enumerate}
	\item Aus Gründen der Übersichtlichkeit werden Variablen und Structs werden nur mit vorangestellten Kommentaren versehen, wenn sie nicht selbsterklärend sind
\end{enumerate}

